\documentclass[a4paper,10pt]{article}
\usepackage[utf8]{inputenc}
\usepackage[T1]{fontenc}
\usepackage[french]{babel}
\usepackage{csquotes}

\usepackage[backend=biber]{biblatex}
\bibliography{bibfile}

\usepackage[hidelinks]{hyperref}

\usepackage{siunitx}
\usepackage{bm}
\usepackage{amsmath}

%opening
\title{PAM Synthèse hybride d'instruments à cordes}
\author{Th\'{e}is Bazin \and Victor Rosi}

\begin{document}

% \maketitle

% \begin{abstract}
% 
% \end{abstract}

\section{Synthèse modale}

\paragraph{}
  On présente dans cette partie le processus d'analyse/synthèse modale hybride
appliqué dans le cadre du projet. Ce processus est basé sur les principes
suivants : les paramètres physiques découplés de corde et de guitare sont
posés, pour la corde via un modèle théorique, pour le corps en appliquant
\textsc{esprit} sur les mesures d'admittance effectuées et en remontant aux
paramètres physiques du système.
  Ensuite, le système d'équations différentielles à \( N \) modes du système
couplé est posé et on résout ce système pour en extraire les déformées modales
et les fréquences propres du système couplé. Enfin, en posant fixant des
conditions initiales, on peut resynthétiser le son de la corde couplée au
chevalet.

\subsection{Paramètres physiques}

\paragraph{}
Par souci de concision, on ne présente pas dans ce rapport le détail des
matrices développées dans l'article de Woodhouse, mais seulement la façon dont
les paramètres physiques pertinents sont extraits ou fixés.

\subsubsection{Corde}
  \paragraph{}
  % INSERT Victor's part.

%   La corde suit un modèle théorique basé sur sa longueur \( L \), sa tension
% \( T \), sa raideur de torsion \( B \) et sa masse linéique \( \rho{} \).
%   On choisit ensuite le nombre de modes de corde désiré.
%   Pour une corde de Mi grave (\( E2 \)), dont la fréquence fondamentale est
% de \( \si{82 \hertz}\), le \( 65 \)ème harmonique aura une fréquence (modulo 
% inharmonicité) proche de \( \si{5330\hertz} \), c'est la valeur maximale que
% l'on fixe, suivant ici les recommandations de Woodhouse.

  Concernant les conditions aux limites, la corde est supposée fixe-fixe
en isolation et le couplage est ensuite réalisé au niveau du chevalet à l'aide
d'un mode de contrainte rigide qui permet d'inclure les vibrations du corps.

\subsubsection{Corps}

  \paragraph{}
  Les paramètres physiques du corps sont extraits par analyse modale via ESPRIT.
Dans l'état actuel, nous ne suivons pas complètement le processus proposé par
Woodhouse, qui ne fait de l'analyse modale que jusqu'à \( \si{1500\hertz} \)
puis, dans la partie haute-fréquences, pose un modèle de répartition modale
statistique.
Notre version souffre d'un overfitting dans les basses-fréquences, on se limite
donc à un nombre restreint de modes de corps pour éviter cet overfitting.

  Une fois les fréquences propres et les amortissements du corps isolé
extraites, on en déduit les paramètres physiques du corps au niveau du
chevalet, modélisé comme un ensemble unidimensionnel de systèmes masse-ressorts
à \( N_b \) modes, d'admittance :
  \[ Y(\omega) = \sum_{k=1}^{N_b} \frac{j\omega{}}
    {m_k(\omega_k^2 + 2 j \omega{} \omega{}_k \xi{}_k - \omega{}^2)} \]
    
Pour ce faire, on définit pour chaque mode calculé une masse effective
\( m_k \) et une raideur effective \( s_k \) ainsi définies :

  Les masses modales sont obtenues par inversion locale autour de
\( \omega = \omega_k \) de l'expression de l'admittance~\cite{pate14:phd} et
valent \[ m_k = \frac{1}{2 |Y(\omega_k)| \omega{}_k \xi{}_k} \] une valeur plus
simple à calculer que celle proposée par Woodhouse -- son expression suppose de
connaître les déformées modales du corps.

  Les raideurs effectives en sont directement déduites et valent
\( s_k = m_k \omega{}_k^2 \), valeur qui permet d'assurer la vibration du système
à la pulsation \( \omega{}_k \).

\subsubsection{Couplage par équations différentielles matricielles}

  \paragraph{}
  L'étape suivante est la définition du système d'équations différentielles
cou\-plées. On se place dans la base des modes propres découplés de corde et de corps,
avec \( N_s \) modes de corde et \( N_b \) modes de corps. On a donc
pour vecteur de coordonnées le vecteur
  \( \bm{q}(t) = [a_1(t), a_2(t), \dots a_{N_s}(t),
    b_1(t), b_2(t), \dots, b_{N_b}(t)] \) de dimension
\( N = N_s + N_b \).
% , où les valeurs \( a_k(t) \) sont les amplitudes (variables
% dans le temps) des \( N_s \) modes de cordes modélisés et les \( b_k(t) \)
% celles des modes de corps retenus.

  L'équation différentielle matricielle du second ordre décrivant le système
couplé soumis à une excitation \( \bm{F} \) s'écrit alors
  \[ \label{eq:diff_second}
    M \ddot{\bm{q}} + C \dot{\bm{q}} + K \bm{q} = \bm{F} \]

  Les valeurs des matrices \( M \) et \( K \), respectivement de masse et de
raideur, sont obtenues dans l'article de Woodhouse par une analyse énergétique
du système et une inversion des résultats obtenus (une sorte de fitting
modal).

  L'amortissement est supposé (hypothèse simplificatrice choisie par Woodhouse)
visqueux (i.e. proportionnel pour chaque mode découplé à sa vitesse) et la
matrice d'amortissement \( C \) est donc diagonale dans la base des
modes découplés.

  Enfin, en supposant \( M \) inversible et en posant le vecteur
\( \bm{p} = \begin{pmatrix} \bm{q} \\ \bm{\dot{q}} \end{pmatrix} \) et la
matrice \( A = \begin{pmatrix} 0 & I \\ -M^{-1}K & -M^{-1}C \end{pmatrix} \)
(entachée d'erreur dans l'article de Woodhouse, ce qui a été la cause de pas
mal de tracas avant que nous ne nous en rendions compte\dots), on réécrit
\ref{eq:diff_second} comme une équation différentielle du premier ordre :
\[ \bm{\dot{p}} = A\bm{p} \]

\paragraph{}
  Les modes propres du système couplé alors obtenus en extrayant les valeurs
propres et une base de vecteurs propres de \( A \).

\section{Synthèse modale}

  La synthèse modale est effectuée selon des formules décrites
dans le livre de~\textcite{newland}, qui traduisent une sommation des modes
pondérée par les amplitudes issues des conditions initiales.

  La condition initiale choisie est actuellement un déplacement triangulaire
de la corde, avec prise en compte d'une largeur de doigt (d'un \( \si{\cm} \))
qui opère comme un filtre passe-bas.

\section{Statut de l'implémentation}

  \emph{TODO : ajouter du quanti / des figures.}
  
  Les résultats sont plutôt satisfaisants à l'oreille avec \( N_s = 40 \) et
\( N_b = 60 \), si l'on augmente \( N_b \), on observe un overfitting
progressif dans les basses fréquences qui concentre de plus en plus d'énergie.
L'amélioration (en cours) de l'utilisation de la méthode \textsc{esprit}
pourrait être la solution à ce problème.

Le véritable point problématique est le temps de calcul très élevé (de l'ordre
de dix minutes pour générer \( 5 \) secondes à \( \si{22050\hertz{}} \))
pour l'étape de resynthèse. Ceci est dû au fait que plusieurs produits
matriciels sont calculés pour chaque sample, au sein d'une boucle for.
L'écriture du calcul de l'ensemble des échantillons en une seule expression
matricielle est difficile en 2D, mais aisée en 3D (le temps devenant la
troisième dimension des matrices).

  L'utilisation d'une bibliothèque MATLAB permettant du calcul rapide sur
des matrices à N-dimensions (un candidat est
\texttt{mmx}~\footnote{%
\url{http://www.mathworks.com/matlabcentral/fileexchange/37515-mmx-multithreaded-matrix-operations-on-n-d-matrices}})
est en cours d'étude et pourrait grandement accélérer le processus.

\printbibliography{}

\end{document}
