\chapter{Méthode ESPRIT - Extraction des paramètres de corde}

\subsection{Objectifs}

Les paramètres physiques et fréquentiels de corde sont nécessaires à la fois à
la synthèse par FRF et à la synthèse modale : dans la première, il est besoin
d'une admittance en extrémité de corde et d'une fonction de transfert entre le
déplacement de deux points de la corde, tandis que la seconde demande la
connaissance des modes de cordes. Pour obtenir l'ensemble de ces données,
plusieurs possibilités s'offrent à nous : s'il est possible de se servir
directement du modèle proposé par l'article de Woodhouse et des valeurs qui y
sont fournies, il pourrait être avantageux d'extraire nous-même ces valeurs de
mesures réelles.\\ Pour cela, il est envisageable de se placer dans une
position "guitariste", en mesurant uniquement la longueur de la corde, sa
tension et sa réponse impulsionnelle en un point. A partir de là, il est
nécessaire de conduire une analyse modale pour extraire de cette réponse les
valeurs des fréquences modales et de leurs amortissements : avec eux, la
synthèse FRF sera assurée. Pour obtenir les données nécessaires à la synthèse
modale, il faudra ensuite "fitter" nos données avec le modèle de Woodhouse
(c.f. partie Victor) afin de déterminer les paramètres $B$, $c$ (et donc
$\rho$), $\eta_A$, $\eta_B$ et $\eta_F$. Un des avantages à effectuer cette
démarche tient ensuite au fait que cette corde pourrait être jouée par nos
méthodes de synthèse en la tendant (i.e. modifiant $T$) ou la raccourcissant
(i.e. modifiant $L$) à volonté, comme une corde d'instrument réel.\\ Procédons
donc comme si nous avions connaissance d'une réponse impulsionnelle de corde,
de sa tension et de sa longueur.

\subsection{Pré-traitement}

Au lieu d'effectuer immédiatement une analyse modale via la méthode ESPRIT
(c.f. un ou deux papiers) de la réponse impulsionnelle, qui risque de se
révéler couteuse en temps, nous utilisons le fait que nous avons déjà une
certaine idée de la forme de l'admittance de la corde et des valeurs de ses
résonances pour effectuer un pré-traitement du signal : afin de n'utiliser la
méthode ESPRIT qu'à faible ordre, nous traitons le signal temporel par un banc
de filtres centrés sur les valeurs approximatives des fréquences attendues. Un
autre intérêt est ensuite de pouvoir effectuer un décalage fréquentiel des
signaux traités (en les modulant par une sinusoïde complexe à la fréquence du
passe-bande correspondant) afin de le décimer et de réduire les temps de calcul
à suivre, tout en permettant aisément de remonter aux valeurs véritables des
pôles du signal. Il est à noter que les signaux en sortie de banc de filtre
sont amputés de leur partie transitoire, qui trahirai le fait qu'il ne s'agit
pas d'exponentielles infinies au yeux de la méthode ESPRIT : pour cela, il est
utile d'utiliser un banc de filtre à réponse impulsionnelle finie (FRI) ,
puisque cette transition contiendra alors précisément autant d'échantillons que
l'ordre des filtres.

\subsection{Méthode ESPRIT et moindres carrés}

- à remplir par une brève explication du principe des méthodes ? -


\subsection{Extractions des paramètres}

Les données obtenues via le traitement précédant permettent immédiatement
l'utilisation de la synthèse de type FRF, mais ne suffisent pas à celle de type
modale. Pour cette dernière, il faut encore remonter aux valeurs de $B$ et de
$\rho$ : pour cela, et comme exposé plus haut, on se sert de l'expression des
valeurs attendues des fréquences et amortissements (c.f. partie Victor) pour
remonter aux paramètres encore inconnus de ces équations. Une fois obtenus, nos
seront en mesure de manipuler ce type de corde à volonté sur nos deux méthodes
de synthèses.



