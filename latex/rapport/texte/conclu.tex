\chapter*{Conclusion et Perspectives}
Lors de ce projet, nous avons eu l'occasion d'implémenter deux méthodes de synthèse hybride différentes, d'établir un protocole de mesures d'admittances, et d'employer des outils de comparaisons basés sur les méthodes HR.

En plus d'une compréhension approfondie du système couplé corde-table d'harmonie, nous obtenons des résultats satisfaisants pour chacune de nos synthèses.

Notre projet peut, à l'aide d'une interface, proposer un outils d'aide à la facture, invitant l'utilisateur à définir les paramètres physiques d'une corde et d'un corps comme la table d'harmonie de la guitare.

Il peut aussi offrir une banque de sons de guitares ou d'autres instruments. L'intérêt étant, à l'aide de la synthèse modale, de pouvoir modéliser des excitations d'amplitudes variables et de répondre avec précision à diverses vélocités.

Quelques pistes (polarisation, rayonnement\dots) resteraient à exploiter : les mesures nécessaires ont été faites (excitation en deux dimensions et prise de son de la table à l'aide d'un micro) il s'agit maintenant d'implémenter leur exploitation. 
\vspace{3mm}
\begin{center}
Enfin, nous tenons à remercier nos encadrants dans le cadre de ce projet :

Bertrand \textsc{David} et Jean-Loïc \textsc{Le~Carrou}.
\end{center}