\section{Analyse/Synthèse modale}

\begin{frame}{Analyse/Synthèse}
 \begin{itemize}
  \item Équation différentielle matricielle
    \[ M \ddot{\bm{q}} + C \dot{\bm{q}} + K \bm{q} = \bm{F} \]
  \item Base des modes propres découplés,
  \item Paramètres des matrices définis:
    \begin{itemize}
     \item Par modèles théoriques (\emph{cf.} Woodhouse, modèle de corde),
     \item Par mesures (ESPRIT sur réponse impulsionnelle de corps).
    \end{itemize}
  \item Réécriture en système du 1er ordre : \( \dot{\bm{z}} = A\bm{z} \), avec
    \( \bm{z} = \begin{pmatrix}
                 \bm{q} \\
		 \dot{\bm{q}}
                \end{pmatrix} \)
  \item Obtention des vecteurs propres et valeurs propres de \( A \),
  \item Projection de la CI sur la base des modes propres couplés,
  \item Somme modale.
 \end{itemize}
\end{frame}

\begin{frame}{Intérêts}
 \begin{itemize}
  \item Flexible : choix du nombre de modes calculés
  \item[\( \implies \)] compromis qualité / temps de calcul,
  \item[\( \implies \)] et haute qualité de synthèse possible.
  \item \emph{Sémantique} : emploi de valeurs pertinentes physiquement
  \item[\( \implies \)] Pratique pour l'aide à la facture.
 \end{itemize}
\end{frame}
